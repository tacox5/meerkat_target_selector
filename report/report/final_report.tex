%% rnaastex.cls is the classfile used for Research Notes. It is derived
%% from aastex61.cls with a few tweaks to allow for the unique format required.
%% (10/15/17)
%%\documentclass{rnaastex}

%% Better is to use the "RNAAS" style option in AASTeX v6.2
%% (01/08/18)
\documentclass[onecolumn]{aastex62}
\usepackage{lipsum}
%% Define new commands here
\newcommand\latex{La\TeX}

%% Tells LaTeX to search for image files in the
%% current directory as well as in the figures/ folder.
\graphicspath{{./}{figures/}}

\begin{document}

\title{Target Selection and Figures of Merit Estimates for the Commensal SETI
       Survey with MeerKAT}

%% Note that the corresponding author command and emails has to come
%% before everything else. Also place all the emails in the \email
%% command instead of using multiple \email calls.
%\correspondingauthor{Tyler Cox}
%\email{greg.schwarz@aas.org, august.muench@aas.org}

\author{Tyler Cox}
\affiliation{School of Earth and Space Exploration, Arizona State University,
             781 Terrace Mall, Tempe, AZ 85287, USA}

\author{Daniel Czech}
\affiliation{Department of Astronomy, University of California, Berkeley,
             501 Campbell Hall \#3411, Berkeley, CA, 94720, USA}

%% The \author command can take an optional ORCID.
\author{David H. E. MacMahon}
\affiliation{Department of Astronomy, University of California, Berkeley,
             501 Campbell Hall \#3411, Berkeley, CA, 94720, USA}

%% Note that RNAAS manuscripts DO NOT have abstracts.
%% See the online documentation for the full list of available subject
%% keywords and the rules for their use.
\keywords{astrobiology ---
technosignature --- SETI --- extraterrestrial intelligence}


\begin{abstract}

One of the flagship missions for the Breakthrough Listen (BL) project is conducting the most comprehensive radio-frequency search for extraterrestrial intelligence (ETI) by observing one million nearby stars. The MeerKAT radio telescope, a precursor to the Square Kilometer Array, will allow BL to pull data streams from primary observations and commensally observe one million stars in the Southern Hemisphere. To organize these observations, we present the MeerKAT Target Selector, a component of the BL commensal survey pipeline to select targets within the field of view of primary observations and rank them based on set criteria. We also present a simulated observational analysis of the one million high-quality targets selected from the Gaia Data Release 2, which will form the basis for further refinement of our target ranking algorithm. Finally, we compare the expected results of our survey to past ETI surveys using modest parameter estimates of the instrument.


\end{abstract}

%% Start the main body of the article. If no sections in the
%% research note leave the \section call blank to make the title.
\section{Introduction}
\lipsum

% Explanation of the pipeline
\section{Breakthrough Listen Backend}
\lipsum

% Target Selector?
\section{Target Selector}
\lipsum

\subsection{Triaging}
\lipsum
\begin{figure*}[ht!]
\plotone{source_field_dist.png}
\caption{Relative figures of merit seti surveys\label{fig:f1}}
\end{figure*}

\begin{figure*}[ht!]
\plotone{example_plot_n_beams.png}
\caption{Relative figures of merit seti surveys\label{fig:f2}}
\end{figure*}

\begin{figure*}[ht!]
\plotone{nbeams_needed.png}
\caption{Relative figures of merit seti surveys\label{fig:f3}}
\end{figure*}

% Results - probably won't be any so change
\section{Survey Figures of Merit}
\lipsum

\begin{equation}
{\rm DFM} = \frac{\Delta \nu_{\rm tot} \Omega}{F_{\rm min}^{3/2}}
\end{equation}

\begin{equation}
{\rm DFM_{\rm tot}} = \sum_{i}^{N} {\rm DFM}_{i}
\end{equation}

\begin{equation}
F_{\rm min} = {\rm S/N_{min}} \frac{2 k_{\rm B} T_{\rm sys}}{A_{\rm eff}} \sqrt{\frac{\rm B}{n_{\rm pol} t_{\rm obs}}}
\end{equation}

\begin{equation}
{\rm CWTFM} = \eta \frac{\rm EIRP_{min}}{N_{\rm star}} \frac{\nu_{c}}{\Delta \nu_{\rm tot}}
\end{equation}

\begin{equation}
{\rm EIRP_{min}} = 4 \pi d^2 F_{\rm min}
\end{equation}


\begin{figure*}[ht!]
\plotone{figure_6_emilio_w_MeerKAT.png}
\caption{Relative figures of merit seti surveys\label{fig:f4}}
\end{figure*}

\begin{figure*}[ht!]
\plotone{figure_5_w_meerkat.png}
\caption{Relative figures of merit seti surveys\label{fig:f5}}
\end{figure*}

\lipsum

%% An example table using AASTeX's deluxetable. Note that since
%% only one figure OR one table is allowed this is commented out.
%\begin{deluxetable}{ccl}
%\tablecaption{Example table some English and Greek letters\label{tab:1}}
%\tablehead{
%\colhead{Index number} & \colhead{English} & \colhead{Greek}
%}
%\startdata
%1 & a & alpha ($\alpha$) \\
%2 & b & beta ($\beta$) \\
%3 & c & gamma ($\gamma$) \\
%4 & d & delta ($\delta$) \\
%5 & e & epsilon ($\epsilon$) \\
%\enddata
%\tablecomments{Long tables should only show a short example with the full
%version as a machine readable table with the article.}
%\end{deluxetable}

%
\section{Conclusions}
asdf asdf asdf

\section{Acknowledgements}
Breakthrough Listen is managed by the Breakthrough Initiatives, sponsored by the Breakthrough Prize Foundation.


%\begin{thebibliography}{}

\bibliographystyle{aasjournal}
\nocite{*}
\bibliography{refs}

%\end{thebibliography}

\end{document}
